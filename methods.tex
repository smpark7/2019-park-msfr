\section{Methodology}

This section discusses the modeling approach and the codes used in this paper.
The reactor model geometry in this paper is kept unchanged from the models
used by Fiorina et al. and Aufiero et al., henceforth referred to as the
Polimi/TUDelft models, to provide a fair comparison in this
benchmarking exercise.

\subsection{Reactor Model Geometry and Group Constant Generation}

For this work, we used the reference square-cylindrical \gls{MSFR} design to
benchmark Moltres against results published by Fiorina et al. and Aufiero et
al. It is a 2-D axisymmetric design with the sixteen individual external loops
homogenized into a single outer loop as shown in Fig. *. For the multi-group
cross sections and group constants calculations in Serpent, we extended this
2-D axisymmetric model into a 3-D model by a simple full rotation about the
central axis. The material definitions are the same as those specified in the
reference \gls{MSFR} model. Accordingly, the pump and heat exchanger regions
are assumed to be composed of 100\% fuel salt. While this may not be entirely
accurate, the exact details of the pump and heat exchanger systems are still
be researched, and this external loop region is presumed to be of little
neutronic importance due to its position behind the strong boron carbide
neutron absorber layer.

There are two key differences between our \gls{MSFR} model geometry in
Moltres and the Polimi/TUDelft models. The first difference is a relative
minor change to the mesh by the exclusion of the 4 cm thick structural
material around the blanket tank that separates the fuel and blanket salts.
We removed this feature in our finite-element mesh as we had difficulty
meshing this layer that is relatively much thinner than the rest of the model.
Any impact on the
neutron flux is expected to be minimal. Furthermore, we solved for the
temperature distribution only in the primary loop and applied homogeneous
Neumann boundary conditions for temperature on the core walls, as was done in
the Polimi/TUDelft models. Therefore, we believe the overall impact on the
results is negligible.

The second difference pertains to the modeling of the external loop. In its
current implementation, Moltres lacks pump- and heat exchanger-equivalents in
the code. Thus, the external loop is modeled as a 1-D pipe with a point heat
sink to represent the heat exchanger. Instead of pumps, the flow is driven by
Dirichlet boundary conditions for velocity on the inlets and outlets of the
2-D axisymmetric central core geometry and the 1-D external pipe geometry.
All flow-dependent variables such as temperature, \glspl{DNP}, and decay heat
precursors are fully conserved as they loop around between the two
regions. As a result, this approach shares some similarities with the
geometric multiscale modeling approach by Zanetti et al. Future models could
create a better representation of the primary loop by implementing a whole
continuous loop with pressure increases and drops corresponding to the pumps
and heat exchangers. 

We used the JEFF-3.1.2 cross-section library for group constant generation in
Serpent. We set a neutron population of 200,000 neutrons per cycle with 50
inactive and 500 active cycles for a total of 100 million neutron histories.
The material compositions of the various reactor components follows the
benchmark specification, with slight adjustments to the
ratio of $^{232}$Th to $^{233}$U to obtain $k_{\text{eff}}=1$ at 973 K. The
resulting mole fraction is very close to a result cited in the neutronics
benchmark paper which was also generated from Serpent using the JEFF-3.1.1
library (Table \ref{table:mole}). The differences can be largely attributed to
the statistical uncertainty for both results. We followed the same six-group
structure as the Polimi COMSOL model \cite{fiorina_modelling_2014}. The
energy boundaries for the six neutron energy groups are shown in Table
\ref{table:bound}. For the \glspl{DNP}, the JEFF-3.1.2 library provides
eight pre-defined \gls{DNP} groups based on their half-lives. The relevant
group constants for Moltres simulations are: the various macroscopic neutron
cross-sections, neutron diffusion coefficient, average fission energy, average
neutron yield, inverse neutron speed, fission spectrum, \gls{DNP} group
constants, and effective delayed neutron fractions.

\begin{table}[h]
	\centering
	\captionsetup{justification=centering}
    \caption{Comparison of mole fractions and $k_{\text{eff}}$ uncertainty
    of $^{232}$Th and $^{233}$U in
    the fuel salt composition adjusted for $k_{eff}=1$ at 973 K.}
\begin{tabular}{l S S}
	\hline
	\textbf{Property} & \textbf{This paper} & \textbf{From benchmark} \\
	\hline
    $^{232}$Th & 19.943 \% & 19.948 \% \\
    $^{233}$U & 2.557 \% & 2.551 \% \\
    $k_{\text{eff}}$ uncertainty & 4.9 pcm & 4.6 pcm \\
    \hline
\end{tabular}
\label{table:mole}
\end {table}

\begin{table}[h]
	\centering
	\captionsetup{justification=centering}
	\caption{Neutron energy group upper bounds used in Serpent.}
	\begin{tabular}{c S}
		\hline
		{Group number} & {Upper bound [MeV]}\\
		\hline
		1 & 7.485$\times 10^{-4}$\\
		2 & 5.5308$\times 10^{-3}$\\
		3 & 2.47875$\times 10^{-2}$\\
		4 & 0.4979\\
		5 & 2.2313\\
		6 & 12\\
		\hline
	\end{tabular}
	\label{table:bound}
\end{table}

library
Composition
energy groups
neutron pop


\subsection{Subsection}

\begin{figure}[h]
        \centering
\begin{tikzpicture}[node distance=1.5cm]
\node (start) [object] {Start};
\node (step1) [process, below of=start] {Step 1};
\node (intermediate) [object, below of=step1] {Intermediate};
\node (step2) [process, below of=intermediate] {Step 2};
\node (end) [object, below of=step2]{End};

\draw [arrow] (start) -- (step1); 
\draw [arrow] (step1) -- (intermediate);
\draw [arrow] (intermediate) -- (step2); 
\draw [arrow] (step2) -- (end);
\end{tikzpicture}
\caption{A caption for the flowchart.}
\label{fig:comp}
\end{figure}
