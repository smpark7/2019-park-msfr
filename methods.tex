\section{Methodology}

This section discusses the modeling approach and the codes used in this paper.

\subsection{Geometry}

For this work, we used the reference square-cylindrical \gls{MSFR} design to
benchmark Moltres against results published by Fiorina et al. and Aufiero et
al. It is a 2-D axisymmetric design with the sixteen individual external loops
homogenized into a single outer loop as shown in figure *. This is the geometry
used in this work to calculate the multi-group cross sections and various
group constants for each material region in Serpent 2. 

\subsection{Subsection}

\begin{figure}[h]
        \centering
\begin{tikzpicture}[node distance=1.5cm]
\node (start) [object] {Start};
\node (step1) [process, below of=start] {Step 1};
\node (intermediate) [object, below of=step1] {Intermediate};
\node (step2) [process, below of=intermediate] {Step 2};
\node (end) [object, below of=step2]{End};

\draw [arrow] (start) -- (step1); 
\draw [arrow] (step1) -- (intermediate);
\draw [arrow] (intermediate) -- (step2); 
\draw [arrow] (step2) -- (end);
\end{tikzpicture}
\caption{A caption for the flowchart.}
\label{fig:comp}
\end{figure}
